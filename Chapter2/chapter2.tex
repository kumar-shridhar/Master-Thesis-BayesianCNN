
%!TEX root = ../thesis.tex
%*******************************************************************************
%****************************** Second Chapter *********************************
%*******************************************************************************

\chapter{Bayesian Neural Network}

\ifpdf
    \graphicspath{{Chapter2/Figs/Raster/}{Chapter2/Figs/PDF/}{Chapter2/Figs/}}
\else
    \graphicspath{{Chapter2/Figs/Vector/}{Chapter2/Figs/}}
\fi

\section{Overview}
\section{Bayesian Inference}
\section{Variational Inference}

Performing Bayesian inference on a neural network requires the posterior distribution of the network weights given the data. If the weights have a prior probability P (w\textbar{}α) that depends on some parameters α, the posterior can be written Pr(w\textbar{}D, α). Unfortunately, for most neural networks Pr(w\textbar{}D, α) cannot be calculated analytically, or even efficiently sampled from. Variational inference addresses this problem by approximating Pr(w\textbar{}D, α) with a more tractable distribution Q (w\textbar{}β). The approximation is fitted by minimising the variational free energy F with respect to the parameters
β, where\\ 
F = − ln  Pr(D\textbar{}w)P (w\textbar{}α) Q (w\textbar{}β) 
w∼Q(β)\\
and for some function g of a random variable x with distribution p(x),
hgix∼p denotes the expectation\\
of g over p. A fully Bayesian approach would infer the prior parameters
α from a hyperprior; however in this paper they are found by simply
minimising F with respect to α as well as β.

\subsection{Monte  Carlo  variational  inference}
\subsection{Local  reparametrisation  trick}

We utilise the local reparameterization trick \cite{kingma2015variational} and apply it to \acp{cnn}. Following \cite{kingma2015variational,neklyudov2018variance}, we do not sample the weights $w$, but we sample instead layer activations $b$ due to its consequent computational acceleration. The variational posterior probability distribution $q_{\theta}(w_{ijhw}|\mathcal{D})=\mathcal{N}(\mu_{ijhw},\alpha_{ijhw}\mu^2_{ijhw})$ (where $i$ and $j$ are the input, respectively output layers, $h$ and $w$ the height, respectively width of any given filter) allows to implement the local reparamerization trick in convolutional layers. This results in the subsequent equation for convolutional layer activations $b$:
\begin{equation}
    b_j=A_i\ast \mu_i+\epsilon_j\odot \sqrt{A^2_i\ast (\alpha_i\odot \mu^2_i)}
\end{equation}
where $\epsilon_j \sim \mathcal{N}(0,1)$, $A_i$ is the receptive field, $\ast$ signalises the convolutional operation, and $\odot$ the component-wise multiplication.

\section{Bayes by Backprop}
\textit{Bayes by Backprop} \cite{graves2011practical, blundell2015weight} is a variational inference method to learn the posterior distribution on the weights $w \sim q_{\theta}(w|\mathcal{D})$ of a neural network from which weights $w$ can be sampled in backpropagation. 
It regularises the weights by minimising a compression cost, known as the variational free energy or the expected lower bound on the marginal likelihood.

Since the true posterior is typically intractable, an approximate distribution $q_{\theta}(w|\mathcal{D})$ is defined that is aimed to be as similar as possible to the true posterior $p(w|\mathcal{D})$, measured by the Kullback-Leibler (KL) divergence \cite{kullback1951information}. Hence, we define the optimal parameters $\theta^{opt}$ as
\begin{equation}
    \begin{aligned} \label{KL}
        \theta^{opt}&=\underset{\theta}{\text{arg min}}\ \text{KL} \ [q_{\theta}(w|\mathcal{D})\|p(w|\mathcal{D})] \\
        &=\underset{\theta}{\text{arg min}}\ \text{KL} \ [q_{\theta}(w|\mathcal{D})\|p(w)] \\ & -\mathbb{E}_{q(w|\theta)}[\log p(\mathcal{D}|w)]+\log p(\mathcal{D})
    \end{aligned}
\end{equation}

where
\begin{equation}
    \text{KL} \ [q_{\theta}(w|\mathcal{D})\|p(w)]= \int q_{\theta}(w|\mathcal{D})\log\frac{q_{\theta}(w|\mathcal{D})}{p(w)}dw .
\end{equation}
This derivation forms an optimisation problem with a resulting cost function widely known as \textit{variational free energy} \cite{neal1998view,yedidia2005constructing,friston2007variational} which is built upon two terms: the former, $\text{KL} \ [q_{\theta}(w|\mathcal{D})\|p(w)]$, is dependent on the definition of the prior $p(w)$, thus called complexity cost, whereas the latter, $\mathbb{E}_{q(w|\theta)}[\log p(\mathcal{D}|w)]$, is dependent on the data $p(\mathcal{D}|w)$, thus called likelihood cost. 
The term $\log p(\mathcal{D})$ can be omitted in the optimisation because it is constant.
\newline Since the KL-divergence is also intractable to compute exactly, we follow a stochastic variational method \cite{graves2011practical,blundell2015weight}.
We sample the weights $w$ from the variational distribution $q_{\theta}(w|\mathcal{D})$ since it is much more probable to draw samples which are appropriate for numerical methods from the variational posterior $q_{\theta}(w|\mathcal{D})$ than from the true posterior $p(w|\mathcal{D})$. Consequently, we arrive at the tractable cost function \eqref{cost} which is aimed to be optimized, i.e. minimised w.r.t. $\theta$, during training:
\begin{equation} \label{cost}
    \mathcal{F}(\mathcal{D}, \theta)\approx \sum_{i=1}^n \log q_{\theta}(w^{(i)}|\mathcal{D})-\log p(w^{(i)})-\log p(\mathcal{D}|w^{(i)})
\end{equation}
%
where $n$ is the number of draws.
\newline We sample $w^{(i)}$ from $q_{\theta}(w|\mathcal{D})$. The uncertainty afforded by \textit{Bayes by Backprop} trained neural networks has been used successfully for training feedforward neural networks in both supervised and reinforcement learning environments \cite{blundell2015weight,lipton2016efficient,houthooft2016curiosity}, for training recurrent neural networks \cite{fortunato2017bayesian}, but has not been applied to convolutional neural networks to-date.

\section{Related Work}
Applying Bayesian methods to neural networks has been studied in the past with various approximation methods for the intractable true posterior probability distribution $p(w|\mathcal{D})$. Buntine and Weigend \cite{buntine1991bayesian} started to propose various \textit{maximum-a-posteriori} (MAP) schemes for neural networks. They were also the first who suggested second order derivatives in the prior probability distribution $p(w)$ to encourage smoothness of the resulting approximate posterior probability distribution. In subsequent work by Hinton and Van Camp \cite{hinton1993keeping}, the first variational methods were proposed which naturally served as a regularizer in neural networks. Hochreiter and Schmidhuber \cite{hochreiter1995simplifying} suggest taking an information theory perspective into account and utilising a minimum description length (MDL) loss. This penalises non-robust weights by means of an approximate penalty based upon perturbations of the weights on the outputs. Denker and LeCun \cite{denker1991transforming}, and MacKay \cite{mackay1995probable} investigated the posterior probability distributions of neural networks by using Laplace approximations. As a response to the limitations of Laplace approximations, Neal \cite{neal2012bayesian} investigated the use of hybrid Monte Carlo for training neural networks, although it has so far been difficult to apply these to the large sizes of neural networks built in modern applications. More recently, Graves \cite{graves2011practical} derived a variational inference scheme for neural networks and Blundell et al. \cite{blundell2015weight} extended this with an update for the variance that is unbiased and simpler to compute. Graves \cite{graves2016stochastic} derives a similar algorithm in the case of a mixture posterior probability distribution. 
\newline Several authors have claimed that Dropout \cite{srivastava2014dropout} and Gaussian Dropout \cite{wang2013fast} can be viewed as approximate variational inference schemes \cite{gal2015bayesian, kingma2015variational}. We compare our results to Gal's \& Ghahramani's \cite{gal2015bayesian} and discuss the methodological differences in detail.

\section{Bayesian Convolutional Neural Network (BayesCNN)}

% Uncomment this line, when you have siunitx package loaded.
%The SI Units for dynamic viscosity is \si{\newton\second\per\metre\squared}.
I'm going to randomly include a picture Figure~\ref{fig:minion}.


If you have trouble viewing this document contact Krishna at: \href{mailto:kks32@cam.ac.uk}{kks32@cam.ac.uk} or raise an issue at \url{https://github.com/kks32/phd-thesis-template/}


\begin{figure}[htbp!] 
\centering    
\includegraphics[width=1.0\textwidth]{minion}
\caption[Minion]{This is just a long figure caption for the minion in Despicable Me from Pixar}
\label{fig:minion}
\end{figure}


\section*{Enumeration}
Lorem ipsum dolor sit amet, consectetur adipiscing elit. Sed vitae laoreet lectus. Donec lacus quam, malesuada ut erat vel, consectetur eleifend tellus. Aliquam non feugiat lacus. Interdum et malesuada fames ac ante ipsum primis in faucibus. Quisque a dolor sit amet dui malesuada malesuada id ac metus. Phasellus posuere egestas mauris, sed porta arcu vulputate ut. Donec arcu erat, ultrices et nisl ut, ultricies facilisis urna. Quisque iaculis, lorem non maximus pretium, dui eros auctor quam, sed sodales libero felis vel orci. Aliquam neque nunc, elementum id accumsan eu, varius eu enim. Aliquam blandit ante et ligula tempor pharetra. Donec molestie porttitor commodo. Integer rutrum turpis ac erat tristique cursus. Sed venenatis urna vel tempus venenatis. Nam eu rhoncus eros, et condimentum elit. Quisque risus turpis, aliquam eget euismod id, gravida in odio. Nunc elementum nibh risus, ut faucibus mauris molestie eu.
 Vivamus quis nunc nec nisl vulputate fringilla. Duis tempus libero ac justo laoreet tincidunt. Fusce sagittis gravida magna, pharetra venenatis mauris semper at. Nullam eleifend felis a elementum sagittis. In vel turpis eu metus euismod tempus eget sit amet tortor. Donec eu rhoncus libero, quis iaculis lectus. Aliquam erat volutpat. Proin id ullamcorper tortor. Fusce vestibulum a enim non volutpat. Nam ut interdum nulla. Proin lacinia felis malesuada arcu aliquet fringilla. Aliquam condimentum, tellus eget maximus porttitor, quam sem luctus massa, eu fermentum arcu diam ac massa. Praesent ut quam id leo molestie rhoncus. Praesent nec odio eget turpis bibendum eleifend non sit amet mi. Curabitur placerat finibus velit, eu ultricies risus imperdiet ut. Suspendisse lorem orci, luctus porta eros a, commodo maximus nisi.

Nunc et dolor diam. Phasellus eu justo vitae diam vehicula tristique. Vestibulum vulputate cursus turpis nec commodo. Etiam elementum sit amet erat et pellentesque. In eu augue sed tortor mollis tincidunt. Mauris eros dui, sagittis vestibulum vestibulum vitae, molestie a velit. Donec non felis ut velit aliquam convallis sit amet sit amet velit. Aliquam vulputate, elit in lacinia lacinia, odio lacus consectetur quam, sit amet facilisis mi justo id magna. Curabitur aliquet pulvinar eros. Cras metus enim, tristique ut magna a, interdum egestas nibh. Aenean lorem odio, varius a sollicitudin non, cursus a odio. Vestibulum ante ipsum primis in faucibus orci luctus et ultrices posuere cubilia Curae; 
\begin{enumerate}
\item The first topic is dull
\item The second topic is duller
\begin{enumerate}
\item The first subtopic is silly
\item The second subtopic is stupid
\end{enumerate}
\item The third topic is the dullest
\end{enumerate}
Morbi bibendum est aliquam, hendrerit dolor ac, pretium sem. Nunc molestie, dui in euismod finibus, nunc enim viverra enim, eu mattis mi metus id libero. Cras sed accumsan justo, ut volutpat ipsum. Nam faucibus auctor molestie. Morbi sit amet eros a justo pretium aliquet. Maecenas tempor risus sit amet tincidunt tincidunt. Curabitur dapibus gravida gravida. Vivamus porta ullamcorper nisi eu molestie. Ut pretium nisl eu facilisis tempor. Nulla rutrum tincidunt justo, id placerat lacus laoreet et. Sed cursus lobortis vehicula. Donec sed tortor et est cursus pellentesque sit amet sed velit. Proin efficitur posuere felis, porta auctor nunc. Etiam non porta risus. Pellentesque lacinia eros at ante iaculis, sed aliquet ipsum volutpat. Suspendisse potenti.

Ut ultrices lectus sed sagittis varius. Nulla facilisi. Nullam tortor sem, placerat nec condimentum eu, tristique eget ex. Nullam pretium tellus ut nibh accumsan elementum. Aliquam posuere gravida tellus, id imperdiet nulla rutrum imperdiet. Nulla pretium ullamcorper quam, non iaculis orci consectetur eget. Curabitur non laoreet nisl. Maecenas lacinia, lorem vel tincidunt cursus, odio lorem aliquet est, gravida auctor arcu urna id enim. Morbi accumsan bibendum ipsum, ut maximus dui placerat vitae. Nullam pretium ac tortor nec venenatis. Nunc non aliquet neque. 

\section*{Itemize}
\begin{itemize}
\item The first topic is dull
\item The second topic is duller
\begin{itemize}
\item The first subtopic is silly
\item The second subtopic is stupid
\end{itemize}
\item The third topic is the dullest
\end{itemize}

\section*{Description}
\begin{description}
\item[The first topic] is dull
\item[The second topic] is duller
\begin{description}
\item[The first subtopic] is silly
\item[The second subtopic] is stupid
\end{description}
\item[The third topic] is the dullest
\end{description}


\clearpage

\tochide\section{Hidden section}
\textbf{Lorem ipsum dolor sit amet}, \textit{consectetur adipiscing elit}. In magna nisi, aliquam id blandit id, congue ac est. Fusce porta consequat leo. Proin feugiat at felis vel consectetur. Ut tempus ipsum sit amet congue posuere. Nulla varius rutrum quam. Donec sed purus luctus, faucibus velit id, ultrices sapien. Cras diam purus, tincidunt eget tristique ut, egestas quis nulla. Curabitur vel iaculis lectus. Nunc nulla urna, ultrices et eleifend in, accumsan ut erat. In ut ante leo. Aenean a lacinia nisl, sit amet ullamcorper dolor. Maecenas blandit, tortor ut scelerisque congue, velit diam volutpat metus, sed vestibulum eros justo ut nulla. Etiam nec ipsum non enim luctus porta in in massa. Cras arcu urna, malesuada ut tellus ut, pellentesque mollis risus.Morbi vel tortor imperdiet arcu auctor mattis sit amet eu nisi. Nulla gravida urna vel nisl egestas varius. Aliquam posuere ante quis malesuada dignissim. Mauris ultrices tristique eros, a dignissim nisl iaculis nec. Praesent dapibus tincidunt mauris nec tempor. Curabitur et consequat nisi. Quisque viverra egestas risus, ut sodales enim blandit at. Mauris quis odio nulla. Cras euismod turpis magna, in facilisis diam congue non. Mauris faucibus nisl a orci dictum, et tempus mi cursus.

Etiam elementum tristique lacus, sit amet eleifend nibh eleifend sed \footnote{My footnote goes blah blah blah! \dots}. Maecenas dapibu augue ut urna malesuada, non tempor nibh mollis. Donec sed sem sollicitudin, convallis velit aliquam, tincidunt diam. In eu venenatis lorem. Aliquam non augue porttitor tellus faucibus porta et nec ante. Proin sodales, libero vitae commodo sodales, dolor nisi cursus magna, non tincidunt ipsum nibh eget purus. Nam rutrum tincidunt arcu, tincidunt vulputate mi sagittis id. Proin et nisi nec orci tincidunt auctor et porta elit. Praesent eu dolor ac magna cursus euismod. Integer non dictum nunc.


\begin{landscape}

\section*{Subplots}
I can cite Wall-E (see Fig.~\ref{fig:WallE}) and Minions in despicable me (Fig.~\ref{fig:Minnion}) or I can cite the whole figure as Fig.~\ref{fig:animations}


\begin{figure}
  \centering
  \begin{subfigure}[b]{0.3\textwidth}
    \includegraphics[width=\textwidth]{TomandJerry}
    \caption{Tom and Jerry}
    \label{fig:TomJerry}   
  \end{subfigure}             
  \begin{subfigure}[b]{0.3\textwidth}
    \includegraphics[width=\textwidth]{WallE}
    \caption{Wall-E}
    \label{fig:WallE}
  \end{subfigure}             
  \begin{subfigure}[b]{0.3\textwidth}
    \includegraphics[width=\textwidth]{minion}
    \caption{Minions}
    \label{fig:Minnion}
  \end{subfigure}
  \caption{Best Animations}
  \label{fig:animations}
\end{figure}


\end{landscape}
